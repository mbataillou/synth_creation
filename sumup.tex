\documentclass[12pt,oneside,a4paper]{article}

\usepackage{./custom}

\begin{document}
% title (fold)
\begin{center}
{\LARGE \bfseries 
 Questions et réponses exams de création d'entreprise \\[0.3cm] 
}
{\large
  \textsc{CentraleSupélec} --- Marc BATAILLOU ALMAGRO --- P2018\\[0.7cm]
}
\end{center}
% title (end)
  
\textbf{\emph{Quels sont les différents stades de financement d’une entreprise, du point de vue du capital risque ? Quels types de risques un investisseur peut il être prêt à prendre ou au contraire à rejeter ?}}

\vspace{1cm}
\begin{tabular}{|c|c|c|} 
	\hline
	 Phase & Montant sollicité & Cycle de vie\\
	 \hline
	 Pré-amorçage & <50K & < début activité comerciale\\
	 \hline
	 Capital amorçage & [50K,250K] & [début acti comerciale, 1.5 ans]\\
	 \hline
	 Capital création & [250K,500K] & [1.5 ans,4 ans]\\
	 \hline
	 Capital développement & [500K, 1000K[ & >4ans\\
	 \hline
\end{tabular} 
\vspace{1cm}

\noindent
\textbf{Pré-amorçage:} prêts familiaux; Prêts d'honneur \\
\textbf{Capital amorçage:} capital de proximité (love money, cigales..); Business Angels ; Incubateurs\\
\textbf{Capital création: }Business Angels ; Venture Corporate\\
\textbf{Capital développement:} Fonds de développement (Venture capital)\\

On peut aussi caractériser les entreprises avec d'autres indicateurs : 

\vspace{1cm}
\begin{center}
\begin{tabular}{|c|c|}
	\hline
	Stade & Risque\\
	\hline
	Concept défini : stade de l'idée & \\
	\hline
	Il existe un prototype : stade de produit & Risque produit\\
	\hline
	Produit industrialisé : stade industriel & Risque industriel\\
	\hline
	Premières ventes : première preuve de marché & Risque de marché\\
	\hline
	Nombreux clients ont acheté : repeat business & Risque d'éxecution\\
	\hline

\end{tabular}
\end{center}
\vspace{1cm}

\textbf{\emph{Peut on parler de relation client / fournisseur entre entrepreneurs et investisseurs en capital? Si oui, ou si non, pourquoi?}}

\textbf{Non}. Les investisseurs \emph{apportent de l’argent à l’entreprise} (et non pas à l’entrepreneur) en participant à une \emph{augmentation de capital}. De ce fait, ils deviennent \textbf{actionnaires} de l’entreprise, au même titre que l’entrepreneur. Ils prennent par ailleurs une \emph{position capitalistique importante} (plusieurs dizaines de pourcents du capital, voire plus de 50$ \%$, au gré des refinancements). Ils négocient très souvent des droits particuliers au niveau de la gouvernance de l’entreprise et ont en quelque sorte un\emph{ `	statut' d’\textbf{associé}}. Ils participent, avec l’entrepreneur, au devenir de l’entreprise, pour \emph{le meilleur et pour le pire}.


\textbf{\emph{Qu’est-ce qu’une clause de buy or sell ? En quoi cette clause est elle importante pour un investisseur ?}}

La clause de buy or sell est une clause \emph{quasi systématiquement présente} dans les pactes d’actionnaires. Elle assure aux investisseurs que les entrepreneurs financés feront leurs meilleurs efforts pour trouver une \textbf{liquidité} à leur investissement, dans un \emph{horizon de temps déterminé à l’avance}. Soit les \emph{entrepreneurs trouvent une liquidité} (sortie) pour les investisseurs, soit ces derniers sont \emph{autorisés à vendre la \textbf{totalité} (des actions)} de l’entreprise.

\textbf{\emph{A quoi et à qui peut servir un Business Model ? Quels sont les éléments nécessaires à sa construction ? }}

Le Business Model modélise le fonctionnement de l’entreprise. Il doit donc comporter les éléments suivants :

\begin{itemize}[label=\ding{69}]
\item la manière dont l’entreprise \underline{produit de la valeur (schéma de production)}
\item la manière dont l’entreprise \underline{accède à ses clients (Go to market)}
\item la manière dont l’entreprise \underline{monétise son offre (Revenue model)}
\end{itemize}

en ayant précisé implicitement :

\begin{itemize}[label=\ding{69}]
\item qui sont les \textbf{clients de l’entreprise (segmentation)}
\item quel est le \textbf{positionnement de l’entreprise dans la chaîne de la valeur}
\end{itemize}

Le Business Model permet à l’entrepreneur \underline{formaliser} puis d’\underline{anticiper} le comportement économique de son entreprise, de \emph{tester des hypothèses}, de \emph{détecter les `sensibilités' ou zones de risque} de son entreprise.
Le Business Model permet aux \emph{investisseurs de mieux appréhender le fonctionnement} de l’entreprise.


\textbf{\emph{Sur quoi le créateur, et plus largement le dirigeant d’une entreprise, doit il garder l’œil rivé «en permanence»? (réponse en 1 mot)}}

\textbf{La trésorerie.}


\textbf{\emph{Quel est le plan type d’un Business Plan ?}}

Le \textbf{Business plan} décrit le projet de création d’entreprise. Il présente notamment :

\begin{itemize}[label=\ding{69}]
	\item L’historique du projet,
	\item Une analyse de marché,
	\item L’offre de l’entreprise et son positionnement par rapports aux offres concurrentes,
	\item Le Business Model de l’entreprise projetée,
	\item L’équipe en place et/ou à construire,
	\item Les chiffres clés passés et à venir,
	\item Le plan de financement.	
\end{itemize}

Le BP est structuré généralement en trois parties : Executive Summary (1 à 2 pages max), Corps du document (8-10 pages) Annexes.

\textbf{\emph{Quelles sont, selon Christophe Crémer, les quatre bases sur lesquelles toute action marketing doit s’appuyer ?}}

\begin{itemize}[label=\ding{69}]
	\item Bénéfices clients
	\item Arguments d’autorité
	\item Incitation à passer à l’acte
	\item Nouveauté	
\end{itemize}








\end{document}